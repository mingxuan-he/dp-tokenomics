\documentclass[12pt]{article}
\usepackage[english]{babel}

% Set page size and margins
% Replace `letterpaper' with`a4paper' for UK/EU standard size
\usepackage[letterpaper,top=2cm,bottom=2cm,left=3cm,right=3cm,marginparwidth=1.75cm]{geometry}

% to typeset URLs, URIs, and DOIs
\usepackage[colorlinks=true, allcolors=blue]{hyperref}
\usepackage{url}
\usepackage{graphicx}
\usepackage[english]{babel}

% Bibliography
\usepackage[style=apa, backend=biber, natbib]{biblatex}
\addbibresource{../references.bib}
%\usepackage{natbib}

% useful packages
\usepackage{amsmath,amsfonts,amssymb,amsthm,mathtools,mathrsfs}
\usepackage{caption,subcaption}
\captionsetup{compatibility=false}
\usepackage{xcolor}
\usepackage{soul}
\usepackage{float}
\def\UrlFont{\rmfamily}

\renewcommand \thesection{\Roman{section}}
\renewcommand \thesubsection{\arabic{section}.\arabic{subsection}}


\begin{document}

\title{Research Project Step 1: Ideas}


\author{Mingxuan He\\
ECMA 33603\\
Prof. Andre Silva}
\date{\today}


\maketitle 

%\tableofcontents

% For those that already have a thesis in progress:
% State the current title of your thesis and explain briefly, in one or two paragraphs, the objective of your thesis.
% List three ideas to add to your current MA project. Explain what you intend to do for each idea. The data that you will need or addition to the model that you will need to develop. Make clear how each idea will improve your current project.

\section*{Thesis topic description}
The title for my thesis is "Dynamic Modeling for Optimal Cryptoeconomic Policies". The objective is to construct a monetary model for cryptocurrency-based economies where the money supply can be adjusted in response to endogenous and exogenous shocks.\\
This research addresses the following gaps in the literature. Most current models feature a fixed or exogenous supply of tokens, which applies to the case for Bitcoin but not for most modern cryptocurrencies. On the other hand, traditional models for optimal fiscal policy (e.g. Ramsey) and monetary policy (e.g. New-Keynsian) are not directly transferrable to cryptoeconomies due to the additional frictions present in the latter. As of today, there has been no established literature on optimal economic policies for cryptocurrencies in general, despite some on CBDCs and stablecoins. \\
I will construct my model in extension to \citeauthor{biais2020equilibrium} (Journal of Finance 2020). They proposed a two-period general equilibrium asset pricing  model for Bitcoin with features including transaction fees as a friction, transactional benefits, and exogenous money supply. In the following sections, I present extensions and/or modifications to this model, most notably attempting to endogenize the money supply via the burning and staking mechanisms.

\section{Two-part transaction fee and burn mechanism}
In an attempt to prevent inflation, many protocols has adopted a burning mechanism by which some cryptocurrencies are permanently removed from supply (``burned"). One approach (e.g. Ethereum) is to have a two-part transaction fee system consisting of a base fee and a variable fee. Mathematically:
\[\varphi_t = \varphi^b_t + \varphi^v_t\]
where the base fee $\varphi^b_t$ is burned. $\varphi^v_t$ is often known as the ``tip" to block validators.\\
In \cite{biais2020equilibrium}, $\varphi_t$ is modeled as a single exogenous variable, and all fees are paid to the validators. Here in the two-part transaction fee system, $\varphi_t^v$ is exogeneously determined by the market for blockchain transactions, while $\varphi_t^b$ is endogenously controlled by the protocol. This allows the protocol to adjust the rate where money supply decreases. The fact that $\varphi^b_t$ also changes the transaction fee on cryptocurrencies is also similar to taxation.\\
The money supply equation with burning:
\begin{equation}
    M_{t+1} = M_t (1 - \varphi^b_t)
\end{equation}


\section{Interest-bearing staking}
Another key mechanism in modern (proof-of-stake) cryptoeconomies is staking rewards. When token holders stake their tokens in the protocol, the protocol rewards them with interest payments proportional to the amount staked. This interest is paid by minting new tokens, and the interest rate is controlled by the protocol. This mechanism serves as the main channel through which new tokens are injected into the economy (increasing money supply), as well as a relatively low-risk investment for token holders. The staked tokens presents a trade-off between liquidity and return to the token holders.\\
The money supply equation with staking:
\begin{equation}
    M_{t+1} = M_t + L_t r_t
\end{equation}
where $L_t$ is the amount of tokens staked in period $t$ and $r_t$ is the nominal interest rate controlled by the protocol.\\
Eqn.1 and 2 can be combined to have a money supply equation with both burning and staking:
\begin{equation}
    M_{t+1} = M_t (1 - \varphi^b_t) + L_t r_t
\end{equation}
The household's maximization problem will change to reflect a new decision between holding cryptocurrency as cash or staking them to earn interest (at the same time avoid being hacked). Note that they must give up transactional benefits (liquidity) in order to do so.


\section{A new method for estimating exogenous variables}
In \cite{biais2020equilibrium}, the amount of transactional benefits agents gain from holder cryptocurrencies is modeled as an exogenous process. In their calibration of the model, the data series on transactional benefits and fees are constructed by counting the events that likely affects the costs and benefits of using bitcoin, and summarize into two indices: \textit{MarketAccess} and \textit{Benefit}. Positive events have a +1 effect on the index, while negative events have a -1 effect.\\
I would like to find a more precise method for estimating these processes. Ideally, I'll be able to find data on the exact transaction fees on a cryptocurrency for a certain time period, to use as the exogenous fee variable. For transactional benefits, I can construct an index using a combination of market share, market capitalization, and number of active accounts (\cite{cong2021tokenomics, cong2022token}) to capture the network effect of the cryptocurrency platform on the transactional benefits and liquidity of the token.



\printbibliography



\end{document}