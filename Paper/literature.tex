\documentclass[./main.tex]{subfiles}
\graphicspath{{\subfix{../Figures/}}}

\begin{document}

\section{Literature Review}
\label{sec:literature}
My research is related to the emerging literature on modeling cryptocurrencies and the blockchain economy. Following \citet{yermack2015bitcoin} that argued that Bitcoin is not a currency, there was an interest in studying Bitcoin as a financial asset. \citet{athey2016bitcoin, garratt2018bitcoin, biais2023equilibrium, pagnotta2022decentralizing} studied equilibria in asset pricing models for Bitcoin. \citet{schilling2019some, schilling2019currency} focused on the interactions between Bitcoin and the real economy, including dollar exchange rate and currency substitution. A broader set of models for Bitcoin include \citet{bolt2020value, catalini2020some, hinzen2022bitcoin, chiu2017economics, huberman2021monopoly}. For proof-of-stake protocols, \citet{saleh2019volatility, saleh2021blockchain} modeled their general volatility and welfare, while \citet{catalini2020markets} examined their tokenomics in the presence of an attack. A common result of the aforementioned asset pricing models is that static token supply causes all shocks to be absorbed by the token price, which explains the high volatility of Bitcoin. This provides the motivation for an endogenous token supply in my model, which adds response mechanisms in terms of dynamic policy rules therefore can potentially mitigate the price effect of exogenous shocks. 

Another set of literature has attempted to identify optimal policy rules for special types of tokens. \citet{cong2021tokenomics, cong2022token} built an asset pricing model featuring token issuance as means of platform financing and user growth leveraging network effects. \citet{sockin2023decentralization, sockin2023model} modeled cryptocurrencies as platform utility tokens and discussed elastic token issuance. \citet{d2022can} studied how stablecoin issuers can profit from seigniorage while maintaining the stablecoin's peg. \citet{fernandez2021central, zhu2019framework} are examples of papers proposing optimal monetary policies for CBDCs. While the policy goals of utility tokens are very specific (defending peg for stablecoins/CBDCs, making profit for stablecoins/platform tokens), I adopt the general policy goal of maximizing user welfare, measured by consumption and transactional benefits.

Two recent papers closely related to mine are \citet{cong2022staking} and \citet{jermann2023macro}. \citet{cong2022staking} applied their dynamic portfolio model to a variety of proof-of-stake blockchains, with a focus on the relationship between equilibrium staking and token carry. Differing from their model, mine features endogenous token supply and emphasizes policy control in response to exogenous shocks. \citet{jermann2023macro} built a macrofinance model for the Ethereum blockchain, characterizing an equilibrium staking ratio that is dependent on transactional benefits and gas fees. It discussed two separate policy tools: functional form of the staking yield and staking yield, while I frame the policymaker problem as adjusting both staking yields and burning rates.

My research is also related to the large literature on cryptocurrencies. Notable empirical researches include \cite{liu2021risks} on the risk factors of cryptocurrencies, \cite{makarov2020trading} on the arbitrage spreads among different cryptocurrency markets. \cite{griffin2020bitcoin} studied the relationship between Tether purchases and cryptocurrency prices. Studies on the microeconomics of blockchain mechanisms include \citet{budish2018economic, biais2019blockchain, gans2019more, gans2022mechanism, huberman2021monopoly}.




\end{document}