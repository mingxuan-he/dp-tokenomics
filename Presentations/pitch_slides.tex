\documentclass{beamer}
\usepackage[utf8]{inputenc}

\usepackage{amsmath,amsfonts,amssymb,amsthm,
mathtools,mathrsfs}
\usepackage{physics}
\usepackage{xcolor}
\usepackage{hyperref}

\usepackage{tikz-cd} % arrow diagram

% Citation style
\usepackage[style=apa, backend=biber, natbib]{biblatex}
\addbibresource{../references.bib}

\usetheme{Madrid}
\usecolortheme{default}
\useoutertheme[subsection=false]{miniframes}
\setbeamertemplate{frametitle continuation}{}
\setbeamertemplate{bibliography item}[triangle]
\setbeamertemplate{sidebar right}{}
%\addtobeamertemplate{footnote}{}{\vspace{2ex}}

% Presenter notes
%\setbeameroption{show notes}



\title[Optimal Cryptoeconomic Policies]{Dynamic Modeling for Optimal Cryptoeconomic Policies}
\subtitle{Pitch Proposal 2}
\author{Mingxuan He}


\institute[UChicago]{
M.A. in Computational Social Science -- Economics\\
Department of Economics, University of Chicago\\
mingxuanh@uchicago.edu
}


\date{\today}


\begin{document}


% title page
\begin{frame}
\titlepage  
\end{frame}

% table of contents
% \begin{frame}
% \frametitle{Table of Contents}
% \tableofcontents
% \end{frame}


%----------------
\section{Introduction}

\begin{frame}{Research Question}

\textbf{How should we design dynamic staking and burning policies?}\newline
\begin{itemize}
\item Current protocol-coded policies are static even though crypto-economies face dynamic shocks. ``Staking" and ``burning" rates are the primary coded policy rules in the underlying blockchain system
\begin{itemize}
    \item \textbf{Staking (increases ``token supply", decreases ``tokens in circulation")}: mint new tokens and give to those who stake their existing tokens as interest. Similar to interest on reserves.
    \item \textbf{Burning (decreases ``token supply")}: burn (remove from circulation) a portion of the transaction fee paid by users during blockchain transactions
    \end{itemize} \textbf{}
    
    \item Policy goal: maximize welfare for \underline{good actors}: users (increase in token price) and validators (transaction fees)
\end{itemize}

\note{
shocks from the real economy: Banks failing, stock market, interest rates\\
shocks from other cryptoeconomies: Luna-Terra, FTX/Alameda
}

\end{frame}

%----------------
%\section{Literature Review}

\begin{frame}{Literature Review}
    \begin{itemize}
        \item Micro foundations:\\
        {\footnotesize \textit{\citet{nisan2007algorithmic, budish2018economic, biais2019blockchain, gans2019more, gans2022mechanism, huberman2021monopoly}, etc.}}\\
        $\rightarrow$ \textbf{Block reward and transaction fee as incentive mechanisms}

        \item Pricing models of cryptocurrencies:\\
        {\footnotesize Bitcoin: \textit{\citet{athey2016bitcoin, garratt2018bitcoin, schilling2019some, schilling2019currency, catalini2020some, biais2020equilibrium, bolt2020value, hinzen2022bitcoin, chiu2017economics}, etc.}}\\
        {\footnotesize Proof-of-stake: \textit{\citet{catalini2020markets, saleh2019volatility, saleh2021blockchain}}}\\
        $\rightarrow$ \textbf{Static token supply means all shocks are absorbed by price}
        \item Monetary policies for stablecoins, platform tokens, and CBDCs:\\
        {\footnotesize \textit{\citet{cong2021tokenomics, cong2022token, d2022can, fernandez2021central, zhu2019framework, sockin2023model, sockin2023decentralization}, etc.}}\\
        $\rightarrow$ \textbf{Optimizing policy for defending peg and/or making profit}
    \end{itemize}
\end{frame}

\begin{frame}{Contribution to the Literature}
\begin{itemize}
    \item Practical: Most cryptoeconomic systems feature fixed-schedule token supply
    \begin{itemize}
        \item deterministic token supply (Bitcoin\footnote{Satoshi Nakamoto (2008). Bitcoin: A peer-to-peer electronic cash system}), 
        fixed burn rate (Ethereum\footnote{Ethereum Documentation. ``Gas and Fees - Base fees"}), 
        naive dynamic burning (Binance Coin)
    \end{itemize}

    \item No published literature has been established on optimal staking and burning policies for proof-of-stake cryptocurrencies.
\end{itemize}
Novelty of this research:
\begin{itemize}
    \item Applies dynamic general equilibrium methods to model and optimize crypto policies for staking and burning            
      \item Traditional models for optimal fiscal policy (e.g. Ramsey) and monetary policy (e.g. NK) are not directly transferrable to cryptoeconomies due to transaction fee role in the cryptoeconomy
  
\end{itemize}
\end{frame}





%----------------
\section{Model (baseline)}
\begin{frame}{The baseline model (Biais et al. 2020, Journal of Finance)}
\textbf{Idea:} The fundamental value of crypto comes from the stream of future transactional benefits \parencite{tirole1985asset}\\
\begin{itemize}
    \item e.g. access to unique goods, not expropriated/taxed/constrained by government, direct internet access
\end{itemize}
\textbf{Setup:} Two-period model with overlapping generations
\begin{itemize}
    \item Three actors: users, validators (miners), hackers
    \item Three financial assets: a risk-free asset, a standard currency (dollar), a cryptocurrency (Bitcoin)
    \item Sources of shocks: endowment, transactional benefit, fees
\end{itemize}
\textbf{Result:} Bitcoin price changes partly due to changes in net transactional benefits, but mostly extrinsic volatility
\end{frame}


\begin{frame}{The baseline model (Biais et al. 2020, Journal of Finance)}
\[
\begin{tikzcd}[ampersand replacement=\&, column sep = 4.5em, row sep = 4.5em]
Protocol 
\ar[d, "\substack{\text{newly minted}\\\text{tokens}}" ']
\&
\ar[d, shift right, "\substack{\text{endow-}\\ \text{ment}}" 'near start]
\ar[d, shift left, "\substack{\text{purchase}\\ \text{assets}}" near start]
\&
\ar[d, "\substack{\text{transact.}\\ \text{benefits}}" ]
\&
\\
Validators
\& 
\substack{Young\\Users}
\ar[l,"\substack{\text{transact.}\\ \text{fees}}" ']
\ar[r, dashed, "t+1", 
"\substack{\text{hold}\\\text{assets}}" ']
\& 
\substack{Old\\Users}
\ar[r, "\substack{\text{hacked}\\ \text{tokens}}"]
\&
Hackers
\end{tikzcd}
\]
In each period: %\hyperlink{formal}{\beamerbutton{Formal Problem}}
\begin{itemize}
    \item \textbf{Young users} receive endowment, spend on consumption goods \& financial assets. Pay fees on crypto purchased
    \item \textbf{Old users} consume savings (some crypto savings get hacked), and receive transactional benefits
    \item \textbf{Validators} receive newly minted tokens as block rewards
    \item \textbf{Hackers} hack a portion of users' crypto
\end{itemize}
\end{frame}


\begin{frame}{The baseline model: Users}
\label{formal}
\begin{align}
    \text{Young users: } c_t^y &= e_t- s_t - (1+\varphi_t) q_t p_t - \hat{q}_t \hat{p}_t\\
    \text{Old users: } c_{t+1}^o &= s_t (1+r_t) + (1-h_{t+1}) q_t p_{t+1} + \hat{q}_t \hat{p}_{t+1} \nonumber \\
    &+ \theta_{t+1} (1-h_{t+1}) q_t p_{t+1} 
\end{align}

$e_t$: endowment\\
$s_t$: quantity of risk-free assets held\\
$q_t, \hat{q}_t$: quantities of crypto and dollars held\\
$p_t, \hat{p}_t$: prices (in units of consumption goods) of crypto and dollars\\
$h_{t+1}$: portion of crypto hacked by hackers\\
$\varphi_t$: transaction fees involved in using crypto (exog.)\footnote{The analysis remain largely unchanged if we include a transaction cost when selling crypto in period $t+1$ as well.}\\
$\theta_{t+1}$: transactional benefits from using crypto (exog.) (assume $\theta_{t+1}\geq-1$) \\

\end{frame}


\begin{frame}{The baseline model: Validators and hackers}
\begin{align}
    \text{Validators: } c_{t+1}^v &= (X_{t+1} - X_t) p_{t+1} + \varphi_{t+1} q_{t+1} p_{t+1}\\
    \text{Hackers: } c_{t+1}^h &= h_{t+1} q_t p_{t+1}
\end{align}
$X_t$: stock token supply\\
$X_{t+1} - X_t$: increase in token supply (newly minted tokens)
\end{frame}

\begin{frame}{The baseline model: Market clearing}
Markets for financial assets:
\begin{align}
    \text{crypto: } q_t &= X_t \\
    \text{dollars: } \hat{q}_t &= m \\
    \text{risk-free assets: } s_t &= 0 \\
\end{align}
Market for consumption goods (by Walras's Law):
\begin{equation}
    c_t^y + c_t^o + c_t^v + c_t^h = e_t
\end{equation}

\end{frame}


\begin{frame}{The baseline model: Solution}
A young user in period $t$ solves:
\begin{align}
\max_{s_t, q_t, \hat{q}_t} &u(c_t^y) + \beta\mathbb{E}_t u(c_{t+1}^o)\\  
    \text{subject to  } & \text{user's budget constraint} \nonumber
\end{align}
$^*$ Information set at period $t$ includes $ \{\theta_t, \varphi_t, \pi_t\}$ \\
From FOCs obtain the equilibrium pricing equation for crypto:
\newcommand{\vp}{\vphantom{\frac{u'(c_{t+1}^o)}{\mathbb{E}_t[u'(c_{t+1}^o)]}}}
\begin{equation}
    p_t = 
    \underbrace{\vp
    \frac{1}{1+r_t}}_\text{discount}
    \mathbb{E}_t \Bigl[ 
    \underbrace{\frac{u'(c_{t+1}^o)}{\mathbb{E}_t[u'(c_{t+1}^o)]}} _\text{risk-neutral prob}
    \underbrace{\vp
    (1-h_{t+1})} _\text{hack risk}
    \underbrace{\vp
    \frac{1+\theta_{t+1}}{1+\varphi_t}} _{\substack{\text{net transact.}\\ \text{benefits}}}
     p_{t+1}
    \Bigr]
\end{equation}    
\end{frame}


%----------------------------------
\section{Model (extended)}
\begin{frame}{Extension 1: Endogenous token supply}
\[
\begin{tikzcd}[ampersand replacement=\&, column sep = 4.5em, row sep = 4.5em]
Protocol 
\ar[d, shift right, "\substack{\text{newly minted}\\\text{tokens}\\(t+1)}" ']
\&
\ar[d, shift right, "\substack{\text{endow-}\\\text{ment}}" 'near start]
\ar[d, shift left, "\substack{\text{purchase}\\ \text{assets}}" near start]
\&
\ar[d, "\substack{\text{transact.}\\ \text{benefits}}" ]
\&
\\
Validators
\ar[u, shift right, red, "\substack{\text{stake}\\\text{tokens}\\(t)}" ' red ]
\& 
\substack{Young\\Users}
\ar[l,"\substack{\text{variable}\\ \text{fees (``tip")}}" ']
\ar[lu, red, "\substack{\text{base fees}\\ \text{(burned)}}" {sloped,auto}' red]
\ar[r, dashed, "t+1", 
"\substack{\text{hold}\\\text{assets}}" ']
\& 
\substack{Old\\Users}
\ar[r, "\substack{\text{hacked}\\ \text{tokens}}"]
\&
Hackers
\end{tikzcd}
\]
I introduce:
\begin{itemize}
    \item Two-part transaction fee with burning
    \item Interest-bearing staking
    \item The protocol controls the staking yield (``interest rate") and the base transaction fee rate (burn rate)
\end{itemize}
\end{frame}


\begin{frame}{Introducing staking and burning}
$^*$ The following is work in progress\\

\textbf{Staking:}
$L_t$: amount of staked tokens; $\delta_t$: staking yield (set by protocol)\\
\textbf{Burning:}
$\varphi_t^b$: base fee (set by protocol); $\varphi_t^v$: variable fee (competitively determined by validators and users)\\
\ \\
\textbf{New budget constraints:}
\begin{align}
    \text{Young users: } c_t^y &= e_t - s_t - (1+{\color{red} \varphi_t^b+\varphi_t^v}) q_t p_t - \hat{q}_t \hat{p}_t\\
    \text{Old users: } c_{t+1}^o &=  s_t (1+r_t) + (1-h_{t+1}) q_t p_{t+1} + \hat{q}_t \hat{p}_{t+1} \nonumber \\
     &\hspace{2em} + \theta_{t+1} (1-h_{t+1}) q_t p_{t+1}\\ 
    \text{Validators: } c_{t+1}^v &= {\color{red} (1+\delta_t) L_t} p_{t+1} + {\color{red} \varphi_{t+1}^v} q_{t+1} p_{t+1} {\color{red} - L_{t+1} p_{t+1} }\\
    \text{Hackers: } c_{t+1}^h &= h_{t+1} q_t p_{t+1}
\end{align}
\end{frame}


\begin{frame}{Token supply with staking and burning}
\begin{align}
    M_{t+1} &= (M_t - L_t) + L_t (1 + \delta_t) - \varphi_t^b X_t \nonumber \\
    &= M_t + \delta_t L_t - \varphi_t^b X_t
\end{align}
$M_t$: total token supply\\
$X_t \equiv M_t - L_t$: circulating supply \\
\quad $\Rightarrow$ Now market clearing condition is $X_t=q_t$\\
\ \\
$^*$ In reality the total reward is $\delta_t L_t^\alpha$, with pre-set $\alpha\in [0,1]$\footnote{$\alpha=0.5$ for Ethereum. See Ethereum Documentation. ``Proof-of-Stake Rewards and Penalties"}. Here I set $\alpha=1$ WLOG since $\delta_t$ is dynamic.
% calculations: https://blog.bitmex.com/ethereums-proof-of-stake-system-calculating-penalties-rewards/
\end{frame}


\begin{frame}{Validator's problem}
% should validators pay transaction fees when they cash out?
\begin{align}
\max_{L_{t+1}}\ & u((1+\delta_t) L_t p_{t+1} + \varphi_{t+1}^v q_{t+1} p_{t+1} - L_{t+1}p_{t+1})\\
    \text{s.t. } & q_{t+1} \leq M_{t+1} - L_{t+1} \nonumber
\end{align}
\textbf{Trade-off:}
\[
\text{more staking at } t \Rightarrow
\begin{cases}
    \text{more staking yield at } t+1\\
    \text{less circulating supply} \Rightarrow \text{less transact. fees at } t
\end{cases}
\]
Apply Karush-Kuhn-Tucker to solve the optimization problem
\end{frame}


%----------------
\section{Next Steps}
\begin{frame}{Next Steps}
    \begin{itemize}
        \item \textbf{The model:} Finish extended model with staking and burning, obtain analytical solution for equilibrium pricing, run simulations using dynare / Python\\
        \item \textbf{Replicate calibration:} Replicate calibration for baseline model (authors provided full replication code in Python)
        \item \textbf{Get data on Ethereum:} 
        \begin{itemize}
            \item ETH price data: Messari API\footnote{https://messari.io/api} (REST) and Dune Analytics\footnote{https://dune.com/docs/api} (SQL database with execution API in Python)
            \item ETH staking and burning data: Beacon Chain\footnote{https://ethereum.github.io/beacon-APIs} (REST)
        \end{itemize}
        \item \textbf{Calibrate my model}: get ideas after replicating calibration
    \end{itemize}
\end{frame}

% \begin{frame}{More extensions to model \& estimation}

% \begin{enumerate}
%     \item Modify validators' budget constraint: incorporates costs of validation under PoS and PoW \parencite{biais2019blockchain, saleh2019volatility, saleh2021blockchain}
%     \item Improvement in model calibration 
%     \begin{itemize}
%         \item novel time series data on Ethereum price and transaction fees
%         \item estimate transactional benefits using the NVT (network value to transaction ratio) or RVT (realized value to transaction ratio) metric instead of event-based index
%     \end{itemize}
%     \item Use vector autoregression models for exogenous process in variable transaction fees and transaction benefit (e.g. sign restrictions)
% \end{enumerate}
% \end{frame}


% \begin{frame}{Proposed Project Timeline}
% \begin{itemize}
%     \item Spring '23: As final project for ECMA33603 (Macro \& Financial Frictions), replicate the baseline model and implement extensions part 1 (endogenous token supply)
%     \item Summer '23: Implement extensions part 2; gather ideas from industry internship in Ethereum research \& dev
%     \item Fall '23 - Winter '24: Extensions part 3: Data collection, model calibration (likely using Ethereum data)
%     \item Spring '24: Finish paper
% \end{itemize}
% \end{frame}


% -----------------
\section*{}
\begin{frame}[allowframebreaks]{References}
    \nocite{*}
    %\bibliographystyle{apalike}
    %\bibliography{references}
    \printbibliography
\end{frame}



\end{document}
